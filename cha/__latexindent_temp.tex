\chapter{概述}
\begin{center}
    \textcolor[RGB]{255, 0, 0}{\faHeart}我记得她很容易哭鼻子,记得她喜欢聊以前的事.记得她在信纸上为我写的每个字,却再也没有见过她一次.\textcolor[RGB]{255, 0, 0}{\faHeart}
\end{center}
\rightline{——《未闻花名》}
\vspace{-5pt}
\begin{center}
    \pgfornament[width=0.36\linewidth,color=lsp]{88}
\end{center}

\section{互联网的组成}
\begin{itemize}
    \item 边缘部分(又称边缘子网):
    
    由所有连接在互联网上的主机组成。这部分是用户直接使用的,
    用来通信(传送数据,音频或者视频)和资源共享
    \item 核心部分(又称通信子网):
    
    有大量网络和连接这些网络的路由器组成。这部分是为边缘部分提供服务的
    (提供连通性和交换)
\end{itemize}
互联网的边缘部分:连接在互联网上的所有的主机,又称为\textcolor{red}{端系统}

\textcolor{green}{小的端系统}:个人电脑,智能手机等

\textcolor{green}{大的端系统}:大型电脑

\textcolor{green}{拥有者}:个人,单位,或某个ISP

\section{互联网的通信}
\subsection{端到端的概念}
\textcolor{red}{主机A和主机B进行通信:}

运行在主机A上的某个程序和运行在主机B上的另一个程序进行通信。即主机A
的某个进程和主机B上的另一个进程进行通信。简称为
\textcolor{blue}{计算机之间通信}

PS:在操作系统中学过同一个计算机的不同的进程之间是可以通信的,但是现在端到端的通信
是在不同的计算机之间通过网络进行通信。

\subsection{俩种通信方式}
\begin{itemize}
    \item \textcolor{red}{客户服务器方式(C/S方式):}
    
    即Client/Server方法,简称为C/S方式

    客户和服务器都是通信中所涉及的俩个应用进程。

    客户/服务器:

    进程之间服务和被服务的关系。客户是服务的\textcolor{red}{请求方},
    服务器是服务的\textcolor{red}{提供方}
    
    \textcolor{red}{客服端软件特点:}

    被调用后运行,在打算通信时主动向远地服务器发起通信(请求服务)。因此,客户程序
    必须知道服务器程序的地址

    不需要特殊的硬件和很复杂的操作系统

    \textcolor{red}{服务器软件特点:}

    一种专门用来提供
    \textcolor{red}{某种服务的程序},可同时处理多个远地或本地客户的请求(例如:IIS,Apache,Profted等)

    启动后一直
    \textcolor{red}{不停地运行着},被动的
    \textcolor{red}{等待并接受}来自客户的
    \textcolor{red}{通信请求},服务器程序不需要知道客户程序的地址。
    
    一般需要
    \textcolor{red}{强大的硬件和高级的操作系统}支持
   
    资源集中在服务器上,优点:数据容易备份,安全性好,缺点:当服务器出问题时就无法提供服务了

    
    
    PS:客服与服务器的通信关系建立后,通信可以是\textcolor{red}{双向的},
    客户和服务器都可\textcolor{red}{发送和接收数据}
   


    \item \textcolor{red}{对等方式(P2P方式):}
    
    即Peer-to-Peer方式,简称为P2P方式

    \textcolor{red}{对等连接:}俩个通信主机
    \textcolor{red}{不区分}哪一个是服务请求方还是服务提供方:

    只要俩个主机都运行了对等连接软件(P2P软件),他们就可以进行平等的、对等连接通信

    对等连接方式仍然是使用客户服务器模式,每一个主机既是客户又是服务器

    对等连接方式可支持
    \textcolor{red}{大量对等用户}(如上百万个)同时工作

    资源分散,一个主机需要的资源可能分散在其他的主机上,他可以就近的先查看与它相连的主机上有无,层层的散发查找

    
\end{itemize}

\section{计算机网络的性能}
\subsection{计算机网络的性能指标}
\begin{itemize}
    \item 速率
    
    \textcolor{red}{速率}是计算机网络中最重要的一个性能指标,指的是数据的传送速
  率,它也称为
  \textcolor{red}{数据率}(data rate)或\textcolor{red}{比特率}(bit rate)。
速率往往是指
\textcolor{red}{额定速率}或
\textcolor{red}{标称速率},非实际运行速率。

\textcolor{red}{速率的单位:}
bit/s(基本单位),或Kbit/s、Mbit/s、Gbit/s等(bps);

例如4 × $10^(10)$bit/s的数据率就记为40 Gbit/s(乘$10^(3)$)。

注意与
\textcolor{red}{存储容量的区别}(乘$2^(10)$)

Byte(字节,基本单位)、KByte(KB)、MByte(MB)、
GByte(GB), 1 Byte = 8 bit。1KB=$2^(10)$B=1024B,

速率是乘以$10^(3)$,而字节是$2^(10)$

    \item 带宽
    
    "带宽"(bandwidth)是指
    \textcolor{red}{信号具有的频率成分范围}。也可表示为
\textcolor{red}{信道的频带宽度}(允许通过的最高频率成分与最低频率成分之差),其
单位是赫(或千赫、兆赫、吉赫等)。

例如:人耳可辨别的声音约在16~20000Hz之间,但只要保留
300~3400Hz这段范围内的声音,仍能清晰辨别话音信号。故电
话信道的带宽为3400-300=3100Hz。


    \item 吞吐量
    \item 时延
    \item 时延带宽
    \item 往返时间
    \item 利用率
\end{itemize}
\subsection{信息的度量}
比特概念:

香农1948年的论文《通信的数学理论》中第一次出现
\textcolor{red}{"比特"}。在香农的理论中,
\textcolor{red}{比特成为信息的基本单位}。回答一个是/否问需要1比
特信息。

比特(bit)也是计算机中
\textcolor{red}{数据量的单位}。比特来源于binary digit,
表示一个“二进制数字”,因此一个比特表示二进制数字中的1或0。


